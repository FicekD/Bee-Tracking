% Soubory musí být v kódování, které je nastaveno v příkazu \usepackage[...]{inputenc}

\documentclass[%        Základní nastavení
%  draft,    				  % Testovací překlad
  12pt,       				% Velikost základního písma je 12 bodů
  a4paper,    				% Formát papíru je A4
  oneside,      			% Jednostranný tisk
	%twoside,      			% Dvoustranný tisk (kapitoly a další důležité části tedy začínají na lichých stranách)
	unicode,						% Záložky a metainformace ve výsledném  PDF budou v kódování unicode
]{report}				    	% Dokument třídy 'zpráva', vhodná pro sazbu závěrečných prací s kapitolami

\usepackage[utf8]		  %	Kódování zdrojových souborů je UTF-8
	{inputenc}					% Balíček pro nastavení kódování zdrojových souborů

\usepackage[				% Nastavení geometrie stránky
	bindingoffset=10mm,		% Hřbet pro vazbu
	hmargin={25mm,25mm},	% Vnitřní a vnější okraj
	vmargin={25mm,34mm},	% Horní a dolní okraj
	footskip=17mm,			  % Velikost zápatí
	nohead,					      % Bez záhlaví
	marginparsep=2mm,		  % Vzdálenost marginálií
	marginparwidth=18mm,	% Šířka marginálií
]{geometry}

\usepackage{sectsty}
	%přetypuje nadpisy všech úrovní na bezpatkové, kromě \chapter, která je přenastavena zvlášť v thesis.sty
	\allsectionsfont{\sffamily}

\usepackage{graphicx} % Balíček 'graphicx' pro vkládání obrázků
											% Nutné pro vložení logotypů školy a fakulty

\usepackage[          % Balíček 'acronym' pro sazby zkratek a symbolů
	nohyperlinks				% Nebudou tvořeny hypertextové odkazy do seznamu zkratek
]{acronym}						
											% Nutné pro použití prostředí 'acronym' balíčku 'thesis'

\usepackage[
	breaklinks=true,		% Hypertextové odkazy mohou obsahovat zalomení řádku
	hypertexnames=false % Názvy hypertext. odkazů budou tvořeny nezávisle na názvech TeXu
]{hyperref}						% Balíček 'hyperref' pro sazbu hypertextových odkazů
											% Nutné pro použití příkazu 'pdfsettings' balíčku 'thesis'

\usepackage{pdfpages} % Balíček umožňující vkládat stránky z PDF souborů
                      % Nutné při vkládání titulních listů a zadání přímo
                      % ve formátu PDF z informačního systému

\usepackage{enumitem} % Balíček pro nastavení mezerování v odrážkách
  \setlist{topsep=0pt,partopsep=0pt,noitemsep} % konkrétní nastavení

\usepackage{cmap} 		% Balíček cmap zajišťuje, že PDF vytvořené `pdflatexem' je
											% plně "prohledávatelné" a "kopírovatelné"

%\usepackage{upgreek}	% Balíček pro sazbu stojatých řeckých písmem
											%% např. stojaté pí: \uppi
											%% např. stojaté mí: \upmu (použitelné třeba v mikrometrech)
											%% pozor, grafická nekompatibilita s fonty typu Computer Modern!
                      
%\usepackage{amsmath} %balíček pro sabu náročnější matematiky                 

\usepackage{dirtree}	% sazba adresářové struktury
                      % vhodné pro prezentaci obsahu elektronické přílohy (např. CD)

\usepackage[formats]{listings}	% Balíček pro sazbu zdrojových textů
\lstset{              % nastavení
%	Definice jazyka použitého ve výpisech
%    language=[LaTeX]{TeX},	% LaTeX
%	language={Matlab},		% Matlab
	language={C},           % jazyk C
    basicstyle=\ttfamily,	% definice základního stylu písma
    tabsize=2,			% definice velikosti tabulátoru
    inputencoding=utf8,         % pro soubory uložené v kódování UTF-8
		columns=fixed,  %fixed nebo flexible,
		fontadjust=true %licovani sloupcu
    extendedchars=true,
    literate=%  definice symbolů s diakritikou
    {á}{{\'a}}1
    {č}{{\v{c}}}1
    {ď}{{\v{d}}}1
    {é}{{\'e}}1
    {ě}{{\v{e}}}1
    {í}{{\'i}}1
    {ň}{{\v{n}}}1
    {ó}{{\'o}}1
    {ř}{{\v{r}}}1
    {š}{{\v{s}}}1
    {ť}{{\v{t}}}1
    {ú}{{\'u}}1
    {ů}{{\r{u}}}1
    {ý}{{\'y}}1
    {ž}{{\v{z}}}1
    {Á}{{\'A}}1
    {Č}{{\v{C}}}1
    {Ď}{{\v{D}}}1
    {É}{{\'E}}1
    {Ě}{{\v{E}}}1
    {Í}{{\'I}}1
    {Ň}{{\v{N}}}1
    {Ó}{{\'O}}1
    {Ř}{{\v{R}}}1
    {Š}{{\v{S}}}1
    {Ť}{{\v{T}}}1
    {Ú}{{\'U}}1
    {Ů}{{\r{U}}}1
    {Ý}{{\'Y}}1
    {Ž}{{\v{Z}}}1
}

\usepackage{amsmath}

%%%%%%%%%%%%%%%%%%%%%%%%%%%%%%%%%%%%%%%%%%%%%%%%%%%%%%%%%%%%%%%%%
%%%%%%      Definice informací o dokumentu             %%%%%%%%%%
%%%%%%%%%%%%%%%%%%%%%%%%%%%%%%%%%%%%%%%%%%%%%%%%%%%%%%%%%%%%%%%%%

% V tomto souboru se nastavují téměř veškeré informace, proměnné mezi studenty:
% jméno, název práce, pohlaví atd.
% Tento soubor je SDÍLENÝ mezi textem práce a prezentací k obhajobě -- netřeba něco nastavovat na dvou místech.

\usepackage[
%%% Z následujících voleb jazyka lze použít pouze jednu
  %czech-english,		% originální jazyk je čeština, překlad je anglicky (výchozí)
  english-czech,	% originální jazyk je angličtina, překlad je česky
  %slovak-english,	% originální jazyk je slovenština, překlad je anglicky
  %english-slovak,	% originální jazyk je angličtina, překlad je slovensky
%
%%% Z následujících voleb typu práce lze použít pouze jednu
  semestral,		  % semestrální práce (nesází se abstrakty, prohlášení, poděkování) (výchozí)
  %bachelor,			%	bakalářská práce
  %master,			  % diplomová práce
  %treatise,			% pojednání o dizertační práci
  %doctoral,			% dizertační práce
%
%%% Z následujících voleb zarovnání objektů lze použít pouze jednu
%  left,				  % rovnice a popisky plovoucích objektů budou zarovnány vlevo
	center,			    % rovnice a popisky plovoucích objektů budou zarovnány na střed (vychozi)
%
]{thesis}   % Balíček pro sazbu studentských prací


%%% Jméno a příjmení autora ve tvaru
%  [tituly před jménem]{Křestní}{Příjmení}[tituly za jménem]
% Pokud osoba nemá titul před/za jménem, smažte celý řetězec '[...]'
\author[Bc.]{Dominik}{Ficek}

%%% Identifikační číslo autora (VUT ID)
\butid{203219}

%%% Pohlaví autora/autorky
% (nepoužije se ve variantě english-czech ani english-slovak)
% Číselná hodnota: 1...žena, 0...muž
\gender{0}

%%% Jméno a příjmení vedoucího/školitele včetně titulů
%  [tituly před jménem]{Křestní}{Příjmení}[tituly za jménem]
% Pokud osoba nemá titul před/za jménem, smažte celý řetězec '[...]'
\advisor[prof.\ Ing.]{Křestní}{Příjmení}[CSc.]

%%% Jméno a příjmení oponenta včetně titulů
%  [tituly před jménem]{Křestní}{Příjmení}[tituly za jménem]
% Pokud osoba nemá titul před/za jménem, smažte celý řetězec '[...]'
% Nastavení oponenta se uplatní pouze v prezentaci k obhajobě;
% v případě, že nechcete, aby se na titulním snímku prezentace zobrazoval oponent, pouze příkaz zakomentujte;
% u obhajoby semestrální práce se oponent nezobrazuje (jelikož neexistuje)
\opponent[doc.\ Mgr.]{Křestní}{Příjmení}[Ph.D.]

%%% Název práce
%  Parametr ve složených závorkách {} je název v originálním jazyce,
%  parametr v hranatých závorkách [] je překlad (podle toho jaký je originální jazyk)
\title[Title of Student's Thesis]{Název studentské práce}

%%% Označení oboru studia
%  Parametr ve složených závorkách {} je název oboru v originálním jazyce,
%  parametr v hranatých závorkách [] je překlad
\specialization[Teleinformatics]{Teleinformatika}

%%% Označení ústavu
%  Parametr ve složených závorkách {} je název ústavu v originálním jazyce,
%  parametr v hranatých závorkách [] je překlad
%\department[Department of Control and Instrumentation]{Ústav automatizace a měřicí techniky}
%\department[Department of Biomedical Engineering]{Ústav biomedicínského inženýrství}
%\department[Department of Electrical Power Engineering]{Ústav elektroenergetiky}
%\department[Department of Electrical and Electronic Technology]{Ústav elektrotechnologie}
%\department[Department of Physics]{Ústav fyziky}
%\department[Department of Foreign Languages]{Ústav jazyků}
%\department[Department of Mathematics]{Ústav matematiky}
%\department[Department of Microelectronics]{Ústav mikroelektroniky}
%\department[Department of Radio Electronics]{Ústav radioelektroniky}
%\department[Department of Theoretical and Experimental Electrical Engineering]{Ústav teoretické a experimentální elektrotechniky}
\department[Department of Telecommunications]{Ústav telekomunikací}
%\department[Department of Power Electrical and Electronic Engineering]{Ústav výkonové elektrotechniky a elektroniky}

%%% Označení fakulty
%  Parametr ve složených závorkách {} je název fakulty v originálním jazyce,
%  parametr v hranatých závorkách [] je překlad
%\faculty[Faculty of Architecture]{Fakulta architektury}
\faculty[Faculty of Electrical Engineering and~Communication]{Fakulta elektrotechniky a~komunikačních technologií}
%\faculty[Faculty of Chemistry]{Fakulta chemická}
%\faculty[Faculty of Information Technology]{Fakulta informačních technologií}
%\faculty[Faculty of Business and Management]{Fakulta podnikatelská}
%\faculty[Faculty of Civil Engineering]{Fakulta stavební}
%\faculty[Faculty of Mechanical Engineering]{Fakulta strojního inženýrství}
%\faculty[Faculty of Fine Arts]{Fakulta výtvarných umění}
%
%Nastavení logotypu (v hranatych zavorkach zkracene logo, ve slozenych plne):
\facultylogo[logo/FEKT_zkratka_barevne_PANTONE_CZ]{logo/UTKO_color_PANTONE_CZ}

%%% Rok odevzdání práce
\graduateyear{2030}
%%% Akademický rok odevzdání práce
\academicyear{2029/30}

%%% Datum obhajoby (uplatní se pouze v prezentaci k obhajobě)
\date{11.\,11.\,1980} 

%%% Místo obhajoby
% Na titulních stránkách bude automaticky vysázeno VELKÝMI písmeny (pokud tyto stránky sází šablona)
\city{Brno}

%%% Abstrakt
\abstract[%
Překlad abstraktu
(v~angličtině, pokud je originálním jazykem čeština či slovenština; v~češtině či slovenštině, pokud je originálním jazykem angličtina)
]{%
Abstrakt práce v~originálním jazyce
}

%%% Klíčová slova
\keywrds[%
Překlad klíčových slov
(v~angličtině, pokud je originálním jazykem čeština či slovenština; v~češtině či slovenštině, pokud je originálním jazykem angličtina)
]{%
Klíčová slova v~originálním jazyce
}

%%% Poděkování
\acknowledgement{%
Rád bych poděkoval vedoucímu diplomové práce panu Ing.~XXX YYY, Ph.D.\ za odborné vedení, konzultace, trpělivost a podnětné návrhy k~práci.
}%  % do tohoto souboru doplňte údaje o sobě, druhu práce, názvu...

%%%%%%%%%%%%%%%%%%%%%%%%%%%%%%%%%%%%%%%%%%%%%%%%%%%%%%%%%%%%%%%%%%%%%%%%

%%%%%%%%%%%%%%%%%%%%%%%%%%%%%%%%%%%%%%%%%%%%%%%%%%%%%%%%%%%%%%%%%%%%%%%%
%%%%%%     Nastavení polí ve Vlastnostech dokumentu PDF      %%%%%%%%%%%
%%%%%%%%%%%%%%%%%%%%%%%%%%%%%%%%%%%%%%%%%%%%%%%%%%%%%%%%%%%%%%%%%%%%%%%%
%% Při načteném balíčku 'hyperref' lze použít příkaz '\pdfsettings':
\pdfsettings
%  Nastavení polí je možné provést také ručně příkazem:
%\hypersetup{
%  pdftitle={Název studentské práce},    	% Pole 'Document Title'
%  pdfauthor={Autor studenstké práce},   	% Pole 'Author'
%  pdfsubject={Typ práce}, 						  	% Pole 'Subject'
%  pdfkeywords={Klíčová slova}           	% Pole 'Keywords'
%}
%%%%%%%%%%%%%%%%%%%%%%%%%%%%%%%%%%%%%%%%%%%%%%%%%%%%%%%%%%%%%%%%%%%%%%%

\pdfmapfile{=vafle.map}

%%%%%%%%%%%%%%%%%%%%%%%%%%%%%%%%%%%%%%%%%%%%%%%%%%%%%%%%%%%%%%%%%%%%%%%
%%%%%%%%%%%       Začátek dokumentu               %%%%%%%%%%%%%%%%%%%%%
%%%%%%%%%%%%%%%%%%%%%%%%%%%%%%%%%%%%%%%%%%%%%%%%%%%%%%%%%%%%%%%%%%%%%%%
\begin{document}
\pagestyle{empty} %vypnutí číslování stránek

%% Vložení desek 
% \includepdf[pages=1]%  buďto generovaných informačním systémem
%   {pdf/student-desky}% název souboru nesmí obsahovat mezery!
%% NEBO vytvoření desek z balíčku
%\makecover
%%
\oddpage % při dvojstranném tisku přidá prázdnou stránku
% kazdopádně ale:
\setcounter{page}{1} %resetovaní čítače stránek -- desky do číslování nezahrnujeme

%% Vložení titulního listu
\includepdf[pages=1]%    buďto generovaného informačním systémem
  {pdf/student-titulka}% název souboru nesmí obsahovat mezery!
%% NEBO vytvoření titulní stránky z balíčku
%\maketitle
%%
\oddpage  % při dvojstranném tisku se přidá prázdná stránka
   
%% Vložení zadání
% \includepdf[pages=1]%   buďto generovaného informačním systémem
%   {pdf/student-zadani}% název souboru nesmí obsahovat mezery!
%% NEBO lze vytvořit prázdný list příkazem ze šablony
%\patternpage{}%
%	{\sffamily\Huge\centering ZDE VLOŽIT LIST ZADÁNÍ}%
%	{\sffamily\centering Z~důvodu správného číslování stránek}
%%
\oddpage% při dvojstranném tisku se přidá prázdná stránka

%% Vysázení stránky s abstraktem
%\makeabstract

% Vysázení stránky s rozšířeným abstraktem
% (pokud píšete práci v češtině či slovenštině, vložení rozšířeného abstraktu zrušte;
%  pro semestrální projekt také není potřeba rozšířený abstrakt uvádět)
% \input{text/rozsireny_abstrakt}

%%% Vysázení citace práce
%\makecitation

%%% Vysázení prohlášení o samostatnosti
%\makedeclaration

%%% Vysázení poděkování
%\makeacknowledgement

%%% Vysázení obsahu
%\tableofcontents

%%% Vysázení seznamu obrázků
% (vynechejte, pokud máte dva nebo méně obrázků)
%\listoffigures

%%% Vysázení seznamu tabulek
% (vynechejte, pokud máte dvě nebo méně tabulek)
%\listoftables

%%% Vysázení seznamu výpisů kódu
% (vynechejte, pokud máte dva nebo méně výpisů)
%\lstlistoflistings

\cleardoublepage\pagestyle{plain}   % zapnutí číslování stránek

%Pro vkládání kapitol i příloh používejte raději \include než \input
%%% Vložení souboru 'text/uvod.tex' s úvodem
\chapter*{Introduction}
\phantomsection
\addcontentsline{toc}{chapter}{Introduction}
The aim of~this work is to~explore solutions for~tracking bees for~the~purposes of~counting the~number of~bees arriving to~and leaving a~hive through a~monitored environment. The~final solution should be applicable on~an~embedded device allowing real-time processing at~a~5 Hz sampling frequency.

In~this work we discuss the~usage of~motion segmentation and other motion tracking techniques for our~task. Our proposed solution stands on~motion signal analysis and is discussed in~the second chapter of~this document.

%%% Vložení souboru 'text/reseni' s popisem řešení práce
% (rozdělte na více souborů či kapitol, pokud je vhodné)
\chapter{Motion tracking techniques}
Our~task stands on~analyzing the~movement of~individual objects, bees traversing through a~predefined environment. There are numerous existing techniques tackling this problem, each with its own sets of~pros and cons. In~this chapter we briefly discuss practical applicability of~several techniques on~our~task.

With the~uprising of~convolutional neural networks and specifically object detectors many works focus on~tracking objects with direct access to detector's predictions on~each frame, for example the Deep SORT \cite{deep_sort} algorithm. This approach is not suitable for our~cause as~convolutional neural networks are computationally expensive.

Another motion analysis technique is to~use initial state of~objects in~their first appearance in~analyzed sequence to~track them though the~sequence using just visual information. A~common approach for~this problem is to~use Siamese neural networks \cite{siamese} that formulate the~problem as convolutional feature cross-correlation between objects and region of~interest. Again this approach requires the use of~computationally expensive neural networks and further more due to~the~large bee traffic though the~observed alley the~initialization step would've had to be ran on~every time step, increasing the~computational cost even more.

Some works focus on~analyzing temporal information though a~low-dimensional subspace of~ambient space with~clustering techniques, these techniques are known as~subspace clustering techniques \cite{subsapce-theory} and are used to~separate low-dimensional data to~their parent subspaces. This approach seems to~be feasible for our~needs at~the~first glance but is not applicable due to~numerous reasons. The most common problem with~subspace clustering variants is that they set requirements on~the~low-dimensional data that are not realistic for~us to~meet, for~example they require prior knowledge of~number of~subspaces \cite{k-subspace,k-flats-subspace} or~prior knowledge of~number of~data points each of~the~subspaces contain \cite{geometric-subspace}. Other approaches generally rely on~building a~similarity matrix from~input data matrix \cite{general-motion-seg,manifold-seg,spectral-clustering}, this approach is also not suitable for~our~needs as~our~low-dimensional data points would be extracted with SIFT-like algorithm, resulting in~inconsistencies in~subsequent data points between time steps, both in~terms of~their location and their quantity. Also, it's worth to~mention that as~the~objects of~interest are bees, which are visually difficult to~distinguish from~each other even for a~human, relying on~local features does not seem like the~correct approach, especially when considering our~low sampling frequency which further lowers the~temporal information each feature's geometric properties carry.

\chapter{Proposed solution}
As~none of~the~previously mentioned motion tracking techniques proved to~be suitable for~our~needs we deviate from~analyzing a~movement of~individual objects of~interest and utilize our~prior knowledge of~environment and analyze only an~overall movement in~region of~interest.

In~order to~keep a~reliable representation of~our~environment even after~a~longer period of~time we utilize dynamic background model. We initialize our~dynamic model $m(x,y,k)$ optionally with~the~first frame of~the~sequence or~with~a~previously captured frame of~the~environment $f(x,y,k=0)$.
\begin{equation}
    m(x,y,0) = f(x,y,0)
\end{equation}
In~each time step of~the~sequence we analyze the~overall dynamic properties of~the~frame with~thresholded subtraction
\begin{equation}
    d_1(x,y,k) = 
    \begin{cases}
        1  & \text{if } \left\lvert f(x,y,k) - f(x,y,k-1) \right\rvert > T_1 \\
        0  & \text{otherwise}
    \end{cases}
\end{equation}
where $T_1$ is empirically selected threshold. And based on~the~overall number of~dynamic pixels we flag the~scene as~either dynamic or~non-dynamic
\begin{equation}
    D_1(k) = \sum _{x,y}d_1(x, y, k) > T_2
\end{equation}
where $T_2$ is again empirically selected threshold. Our~dynamic model is updated only when the~scene is flagged as~static in~sequence, meaning $D_1$ is flagged as~\textit{false}, and the~scene is flagged as~static with~respect to~the~dynamic model. This flag is calculated in~similar fashion as~the~$D_1$ flag.
\begin{equation}
    d_2(x,y,k) = 
    \begin{cases}
        1  & \text{if } \left\lvert f(x,y,k) - m(x,y,k-1) \right\rvert > T_1 \\
        0  & \text{otherwise}
    \end{cases}
\end{equation}
\begin{equation}
    D_2(k) = \sum _{x,y}d_2(x, y, k) > T_2
\end{equation}
The~dynamic model is then updated with simple adaptive filter
\begin{equation}
    m(x, y, k) = 
    \begin{cases}
        \alpha \cdot f(x,y,k) + (1-\alpha)\cdot m(x, y, k-1)  & \text{if } D1(k) \text{ and } D2(k) \\
        m(x, y, k-1)  & \text{otherwise}
    \end{cases}
\end{equation}
where $\alpha$ is the~learning rate of~the~dynamic model.

To~track the~level of~dynamic activity in~our~region of~interest, we inspect the~ratio of~dynamically flagged pixels in~subtracted frame from~the~current time step with~our~dynamic environment model to~the~number of~pixels in~the~region. However, this metric alone does not carry any information in terms of~the~movement's direction. To~compensate this we split the~inspected region into~$N$ sections along the~$y$ axis.
\begin{equation}
    \begin{gathered}
        f(x,y) \rightarrow g(x,y,n) \\
        \rm I\!R^{X\times Y} \rightarrow \rm I\!R^{X\times \left\lfloor \frac{Y}{N}\right\rfloor  \times N}
    \end{gathered}
\end{equation}
In~each of~these $N$ sections we calculate the~already mentioned metric of~dynamic activity
\begin{equation}
    d(x,y,n,k) = 
    \begin{cases}
        1  & \text{if } \left\lvert f(x,y,n,k) - m(x,y,n,k) \right\rvert > T_1 \\
        0  & \text{otherwise}
    \end{cases}
\end{equation}
\begin{equation}
    r(n, k) = \frac{1}{X \cdot Y} \sum _{x,y} d(x,y,n,k)
\end{equation}
this $r(n,k)$ signal now carries enough information to~determine the~level and direction of~movement in~observed region. To~further ease this signal's processing we approximate first order partial derivative with~respect to~the~time step dimension with~differentiation
\begin{equation}
    dr(n, k) = r(n, k) - r(n, k-1)
\end{equation}
In~the~resulting signal $dr(n, k)$ we threshold its peaks to~classify the~current timestep with~one of~three classes: bee arrival ($class = 1$), idle state ($class = 0$) and bee departure ($class = -1$).
\begin{equation}
    class(n, k) = 
    \begin{cases}
        1  & \text{if } dr(n, k) > T_3 \\
        -1  & \text{if } dr(n, k) < -T_3 \\
        0  & \text{otherwise}
    \end{cases}
\end{equation}
where $T_3 \in \left\langle 0, 1\right\rangle $ is empirically selected threshold value.

To~implement the~actual counter of~bees that traverse the~region in~one way or~the~other we keep track of~classes from~the~last $K_{max}$ time steps. On~each bee arrival we add a~track to~a~list and with~each bee departure we flag an~unflagged track as~valid. Once a~valid track is present in~all $N$ sections, we increment a~counter. The~direction of the~bee's movement is based on~the~age of~the~valid tracks on~the~edges of~the~region.

As~this is quite a~simple approach, it's bound to~have limitations, its main disadvantage is that if the bees do not travel independently but~in~packs, $r(n,k)$ will remain close to~constant, $dr(n, k)$ will be close to~zero and the~bees won't be accounted for. Other limitation that we have observed in~experiments is that once a~bee slows down at~some point of~its traverse through the~observed area or the~lightning in~the observed area lowers its intensity, the~$dr(n, k)$ values reduces sometimes even to~a~noise level. This leads to~a~missing track entry in~one or~more sections of~a~tunnel, the~bee is not accounted for~and there may be hanging tracks left in~sections where the~bee was registered. This effect can have detrimental effect on~next registered bees as~hanging tracks on~the~edges of~the~observed region define estimated traverse direction. This effect can be suppressed lowering the~maximum track age $K_{max}$ but~lowering this value also effectively reduces sensitivity when the~bee's velocity lowers. On~the~bright side this approach runs under~20ms on~Raspberry Pi 4B, sequentially processing 12 bee tunnels in~each frame.

%%% Vložení souboru 'text/vysledky' s popisem vysledků práce
% (rozdělte na více souborů či kapitol, pokud je vhodné)
% \include{text/vysledky}

%%% Vložení souboru 'text/zaver' se závěrem
\chapter*{Conclusion}
\phantomsection
\addcontentsline{toc}{chapter}{Conclusion}

In~this work we tackle a~problem of~counting bees leaving and arriving to~a~hive. After~a~brief summary of~motion tracking techniques in~the~first chapter we decided not to~use any of~them as~they are not very suited for our needs, we describe our~solution in~the~second chapter. The solution lies in~analyzing overall movement in~multiple sections of~region of~interest to~detect general movement and determine its direction. The algorithm's limitations are discussed at~the~end of~the~second chapter, and it can be run on~an~embedded device as~a~part of~online processing pipeline.

%%% Vložení souboru 'text/literatura' se seznamem zdrojů
% Pro sazbu seznamu literatury použijte jednu z následujících možností

%%%%%%%%%%%%%%%%%%%%%%%%%%%%%%%%%%%%%%%%%%%%%%%%%%%%%%%%%%%%%%%%%%%%%%%%%
%1) Seznam citací definovaný přímo pomocí prostředí literatura / thebibliography

\begin{thebibliography}{99}

\bibitem{deep_sort}
N. Wojke, A. Bewley and D. Paulus, "Simple online and realtime tracking with a deep association metric," 2017 IEEE International Conference on Image Processing (ICIP), 2017, pp. 3645-3649, doi: 10.1109/ICIP.2017.8296962.

\bibitem{siamese}
B. Li, W. Wu, Q. Wang, F. Zhang, J. Xing and J. Yan, "SiamRPN++: Evolution of Siamese Visual Tracking With Very Deep Networks," 2019 IEEE/CVF Conference on Computer Vision and Pattern Recognition (CVPR), 2019, pp. 4277-4286, doi: 10.1109/CVPR.2019.00441.

\bibitem{subsapce-theory}
E. Elhamifar and R. Vidal, "Sparse Subspace Clustering: Algorithm, Theory, and Applications," in IEEE Transactions on Pattern Analysis and Machine Intelligence, vol. 35, no. 11, pp. 2765-2781, Nov. 2013, doi: 10.1109/TPAMI.2013.57.

\bibitem{k-subspace}
J. Ho, Ming-Husang Yang, Jongwoo Lim, Kuang-Chih Lee and D. Kriegman, "Clustering appearances of objects under varying illumination conditions," 2003 IEEE Computer Society Conference on Computer Vision and Pattern Recognition, 2003. Proceedings., 2003, pp. I-I, doi: 10.1109/CVPR.2003.1211332.

\bibitem{k-flats-subspace}
Teng Zhang, A. Szlam and G. Lerman, "Median K-Flats for hybrid linear modeling with many outliers," 2009 IEEE 12th International Conference on Computer Vision Workshops, ICCV Workshops, 2009, pp. 234-241, doi: 10.1109/ICCVW.2009.5457695.

\bibitem{geometric-subspace}
Sugaya, Yasuyuki and Kenichi Kanatani. "Geometric structure of degeneracy for multi-body motion segmentation." International Workshop on Statistical Methods in Video Processing. Springer, Berlin, Heidelberg, 2004.

\bibitem{general-motion-seg}
Yan, Jingyu and Marc Pollefeys. "A general framework for motion segmentation: Independent, articulated, rigid, non-rigid, degenerate and non-degenerate." European conference on computer vision. Springer, Berlin, Heidelberg, 2006.

\bibitem{manifold-seg}
A. Goh and R. Vidal, "Segmenting Motions of Different Types by Unsupervised Manifold Clustering," 2007 IEEE Conference on Computer Vision and Pattern Recognition, 2007, pp. 1-6, doi: 10.1109/CVPR.2007.383235.

\bibitem{spectral-clustering}
Ng, Andrew, Michael Jordan and Yair Weiss. "On spectral clustering: Analysis and an algorithm." Advances in neural information processing systems 14 (2001).

\end{thebibliography}


%%%%%%%%%%%%%%%%%%%%%%%%%%%%%%%%%%%%%%%%%%%%%%%%%%%%%%%%%%%%%%%%%%%%%%%%%
%%2) Seznam citací pomocí BibTeXu
%% Při použití je nutné v TeXnicCenter ve výstupním profilu aktivovat spouštění BibTeXu po překladu.
%% Definice stylu seznamu
%\bibliographystyle{unsrturl}
%% Pro českou sazbu lze použít styl czechiso.bst ze stránek
%% http://www.fit.vutbr.cz/~martinek/latex/czechiso.tar.gz
%%\bibliographystyle{czechiso}
%% Vložení souboru se seznamem citací
%\bibliography{text/literatura}
%
%% Následující příkaz je pouze pro ukázku sazby literatury při použití BibTeXu.
%% Způsobí citaci všech zdrojů v souboru literatura.bib, i když nejsou citovány v textu.
%\nocite{*}

%%% Vložení souboru 'text/zkratky' se seznam použitých symbolů, veličin a zkratek
% \include{text/zkratky}

%%% Začátek příloh
%\appendix

%%% Vysázení seznamu příloh
% (vynechejte, pokud máte dvě nebo méně příloh)
%\listofappendices

%%% Vložení souboru 'text/prilohy' s přílohami
% Obvykle je přítomen alespoň popis co najdeme na přiloženém médiu
% \include{text/prilohy}

\end{document}